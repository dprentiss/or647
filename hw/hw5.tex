\documentclass[letterpaper]{amsart}
\usepackage{float}
\usepackage{times}
\usepackage{tikz}
\usepackage{calc}
\usepackage{booktabs, siunitx}
\usepackage{graphicx}
\usepackage{mathtools}
\usetikzlibrary{arrows.meta}

\title[Homework 5]{Homework 5 \\ OR647: Queueing Theory, Spring 2021}
\author{David Prentiss}
\email{dprentis@gmu.edu}
\date{\today}

\begin{document}
\maketitle

\section{Problem 5.23} %1
Find the average number of customers at each node and the node delay time for a closed network with 35 customers circulating between seven nodes using the switch matrix given below and assuming the same service rates as in Problem 5.10.
\subsubsection*{Problem 5.10}
Nodes 1 and 2 have service rates of 85, nodes 3 and 4 have rates of 120, node 5
has a rate of 70, and nodes 6 and 7 have rates of 20 (all in units per minute).
\begin{equation*}
  \begin{bmatrix}
    \frac{1}{3} & \frac{1}{4} & 0 & \frac{1}{4} & 0 & \frac{1}{6} & 0 \\[0.5ex]
    \frac{1}{3} & \frac{1}{4} & 0 & \frac{1}{4} & \frac{1}{6} & 0 & 0 \\[0.5ex]
    0 & 0 & \frac{1}{3} & \frac{1}{3} & \frac{1}{3} & 0 & 0 \\[0.5ex]
    \frac{1}{3} & 0 & \frac{1}{3} & 0 & \frac{1}{3} & 0 & 0 \\[0.5ex]
    0 & 0 & 0 & \frac{5}{6} & 0 & 0 & \frac{1}{6} \\[0.5ex]
    \frac{1}{6} & \frac{1}{6} & \frac{1}{6} & \frac{1}{6} & \frac{1}{6} & \frac{1}{6} & 0 \\[0.5ex]
    0 & \frac{1}{6} & \frac{1}{6} & \frac{1}{6} & \frac{1}{6} & \frac{1}{6} & \frac{1}{6}
  \end{bmatrix}
\end{equation*}
\subsubsection*{Solution}
Let $M=35$ and $\mu=\{85,85, 120, 120,70, 20, 20\}$.
Solving the traffic equations with $v_6=1$ we have
\begin{equation*}
v\approx\{5.536, 2.358, 4.607, 8.060, 5.000, 1.307, 1.000\}
\end{equation*}
Then, using the Bard-Schweitzer algorithm for mean-value anaylsis, we find
\begin{equation*}
  L(35) \approx \{5.635, 0.579, 1.023, 7.018, 13.073, 5.767, 1.904\}
\end{equation*}
Then, using mean-value anaylsis again, we find
\begin{equation*}
  L(34) \approx \{5.519, 0.577, 1.017, 6.843, 12.511, 5.646, 1.887\}
\end{equation*}
So then,
\begin{equation*}
  W(35) = \left\{\frac{1+L_i(34)}{\mu_i}\right\}_{i\in\{1,2,\dots,7\}}\approx \{0.077, 0.019, 0.017, 0.065, 0.193, 0.332, 0.144\}
\end{equation*}



\section{Problem 5.24} %2
Tonyexpress, a regional truck delivery service has a fleet of 50 trucks.
Routine maintenance is done during off hours, so no trucks are unavailable
for service because of routine maintenance. However, trucks do break
down periodically and require repair. The breakdown process can be well
approximated by a Poisson process, and repair records indicate a mean
time to breakdown of 38 days. Sixty-eight percent of these breakdowns
can be handled at Tonyexpress’ own repair facility; the remainder must be
handled by the manufacturer (the fleet consists of all the same make of
trucks). Furthermore, about 7\% of those sent to local repair must, after
being worked on, be sent on to the manufacturer’s repair facility. Repair
times at the local repair facility are exponentially distributed with a mean
of 2.75 days, and there are four repair bays, but only three operate at any
given time, since there are only three repair crews available. Turnaround
times for those sent to the factory have been found to be IID exponential
random variables, with a mean of 10 days (since Tonyexpress is such a good
customer, whenever a truck is received by the manufacturer, a mechanic
is immediately put on the job). Since trucks are so well cared for, it is
not unreasonable to assume that repaired trucks are as good as new. When
trucks break down, Tonyexpress has a contract with TowsRus, a towing
company with a very ample fleet of tow trucks. The mean time to pick up
and tow a truck to the Tonyexpress facility is 0.15 days, and the mean time
to tow a truck to the manufacturer (from the field or from Tonyexpress) is
0.75 days. Tony, the CEO of Tonyexpress, hires you to advise whether he
should hire another repair crew. His major performance measure is truck
availability he is interested in the expected fraction of trucks available and
the percentage of time that he has at least 45 trucks operational.
\subsubsection*{Solution}
(I did not get QTS running and I ran out of time implementing the regular
mean-value analysis algorithm in Python. There is some work in the attched code,
for what it's worth.)


\section{Problem 4.4} %3
Shuttle buses take customers from an airport terminal to a rental-car agency.
Suppose that the arrival of buses at the rental-car agency is a Poisson process
with rate 6 per hour. The number of passengers on a given bus roughly
follows a Poisson random variable with a mean of 5.
\subsection*{a}
Suppose that there is a single server at the rental-car agency and that the
time for the server to process a customer is exponentially distributed
with a mean of 1.5 min. Determine the average time each customer
spends in the rental-car agency (including service time).
\subsubsection*{Solution}
Assume $M^{[X]}/M/1$ model.
Let $\lambda = 6/60 = 1/10 \text{ per/min}$, $\text{E}[X] = 5$ and $\mu=2/3\text{ per/min}$.
Then $\rho = 1/10\cdot 5 \cdot 3/2 =3/4$, $\text{E}[X^2] = \text{Var}[X] +
(\text{E}[X])^2=5 + 25 = 30$ and
\begin{equation*}
  L = \frac{3/4+1/10\cdot 3/2\cdot 30}{2(1-3/4)} = 39/4 = 9.75.
\end{equation*}
So,
\begin{equation*}
  W = \frac{L}{\lambda\text{E}[X]}=39/2 = 19.5 \text{ min}.
\end{equation*}
\subsection*{b}
Now, suppose that the rental-car agency is located right next to the
terminal, so that customers can walk directly to the agency rather than
taking a shuttle. In this case, it may be reasonable to assume that
customers arrive according to a Poisson process with a rate of 30 per
hour. Now, what is the average time each customer spends in the rental
car agency?
\subsubsection*{Solution}
Assume $M/M/1$ model.
Let $\lambda = 1/2\text{ per/min}$ and $\mu = 2/3\text{ per/min}$.
Then $\rho = 3/4$ and
\begin{equation*}
  W = \frac{1}{\mu - \lambda} =\frac{1}{2/3 - 1/2}= 6\text{ min}.
\end{equation*}

\section{Problem 4.22} %4
Isle-Air Airlines offers air shuttle service between San Juan and Charlotte
Amalie every 2 h. The procedure calls for no advance reservations but
for passengers to purchase their tickets at the gate from which the shuttle
leaves. It is found that passengers arrive according to a Poisson distribution
with a mean of 18/h. There is one agent at the gate check-in counter, and
a time study provided the following 50 observations on the processing time
in minutes: 4.00, 1.44, 4.44, 1.74, 1.16, 4.20, 3.59, 2.14, 3.54, 2.56, 5.53, 2.02, 3.06,
1.66, 3.23, 4.84, 7.99, 3.07, 1.24, 3.40, 5.01, 2.78, 1.62, 5.19, 5.09, 3.78,
1.52, 3.94, 1.96, 6.20, 3.67, 3.37, 1.84, 1.60, 1.31, 5.64, 0.99, 3.06, 1.24,
3.11, 4.57, 0.90, 2.78, 1.64, 2.43, 5.26, 2.11, 4.27, 3.36, 4.76.

On average, how many are in the queue waiting for tickets, and what is the
average wait in the queue? [Hint: Find the sample mean and variance of
the observed service times, and see what distribution might fit.]
\subsubsection*{Solution}
The sample mean and variance are approximately 3.197 and 2.464 respectively.
Assume the distribution of service times is $E_4(5/4)$ with mean and
variance of $16/5=3.2$ and $16/25=2.56$ respectively.
Assume also an $M/E_4/1$ model.
Let $\lambda=18/60=3/10$.
Then $\rho = 24/25$ and we have
\begin{align*}
  L_q &= \frac{1+1/k}{2}\frac{\rho^2}{1-\rho}
        = \frac{1+1/4}{2}\frac{(24/25)^2}{1-24/25}
  = \frac{72}{5} = 14.4\\
  W_q &= \frac{L_q}{\lambda} = \frac{72}{5}  \frac{10}{3}= 48\text{ min}.
\end{align*}

\section{} %4
Provide a short paragraph description of what you plan to do for your project.
\subsubsection*{Solution}
See attached PDF. This is what I have so far. There is no introduction at the
moment but I hope to send that along soon.
\end{document}