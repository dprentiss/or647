\documentclass[letterpaper]{amsart}
\usepackage{cancel}
\title[Homework 1]{Homework 1 \\ OR647: Queueing Theory, Spring 2021}
\author{David Prentiss}
\email{dprentis@gmu.edu}
\date{\today}
\begin{document}
\maketitle

\section{} %1
A certain football league consists of 32 teams. Each team has 67 active
players. There is a draft each year for teams to acquire new players. Each
team acquires 7 new players per year in the draft. The number of active
players on each team must always be 67. Thus, each team must cut some
existing players each year to make room for the new players.

\subsection*{a}
Assuming that a football player can only join a team by being selected
in the draft, estimate the average career length of a football player in
the league.


\subsubsection*{Solution}

From Little's Law, we know that $L=\lambda W$, where $L$ is the average number
of customers in the system, $\lambda$ is the average rate that customers arrive
to a system, and $W$ is the average time that a customer spends in the system.

Let the system be a single team. Since the number of active
players per team must be 67, then let $\L = 67 \text{ players}$. Furthermore, since seven
new players are drafted each year, let $\lambda = 7 \text{ players/year}$.

Solving the equation of Little's Law for $W$ we have
\begin{align}
  L &= \lambda W \\
  W &= \frac{L}{\lambda} = \frac{67 \text{ players}}{7 \text{ players/year}} \approx 9.6 \text{ years}.
\end{align}
Note that if we take the system to be the entire league, then
$\L = 67 \text{ players} \times 32 \text{ teams}$
,
$\lambda = 7 \text{ players/year}\times 32 \text{ teams}$
, and we get the same result with
\begin{equation}
  W = \frac{L}{\lambda} = \frac{67 \text{ players} \times \cancel{32 \text{ teams}}}
  {7 \text{ players/year}\times \cancel{32 \text{ teams}}} \approx 9.6 \text{ years}.
\end{equation}

\subsection*{b}
Now, suppose that a player can join a team in one of two ways: (1) by
being selected in the draft, as before, or (2) by signing directly with a
team outside the draft. Suppose further that the average career length of
a football player is known to be 3.5 years. Under the same assumptions
as before, estimate the average number of players who enter the league
each year without being drafted.
\subsubsection*{Solution}
We are given $W = 3.5 \text{ years}$. Additionally, the average
number of players joining a team each year is the sum of the seven players
drafted plus the average number of players signing on outside of the draft,
$\lambda_d$, giving $\lambda = 7+\lambda_d$. Solving again, this time for $\lambda_d$, we have
\begin{align}
  L &= \lambda W \\
  L &= (7 + \lambda_d) W \\
  \lambda_d &= \frac{L}{W}-7 = \frac{67 \text{ players}}{3.5 \text{ years}}
              - 7\text{ players/year}\approx 12.1\text{ players/year}.
\end{align}

\section{} %2
The length of time that a person owns a car before buying a new one has an Erlang-3
distribution with a mean of 5 years. Suppose that there are approximately 150 million
cars in the United States.

\subsection*{a}
Assuming that a person's old car is destroyed when he or she buys a new car,
how many cars does the auto industry expect to sell each year?
\subsubsection*{Solution}
Applying Little's Law, let the system be the pool of cars in the United States
with the average number of cars in the system, $L=1.5\times10^8\text{ cars}$, and the
average time spent in the system, $W=5 \text{ years}$. Then, solving for the
average rate of cars entering the the system, $\lambda$, we have
\begin{align}
  L &= \lambda W \\
  \lambda &= \frac{L}{W} = \frac{1.5\times10^8\text{ cars}}{5 \text{ years}} = 3\times10^8\text{ cars/year}
\end{align}
\subsection*{b}
Now assume that a person's old car is sold to somebody else when that person
buys a new car. The person who buys the used car keeps it for period of time
following an Erlang-3 distribution with a mean of 7 years. When that person buys
another used car, his or her previous used car is assumed to be destroyed. Under
the same previous assumptions, how many new cars does the auto industry
expect to sell each year?
\subsubsection*{Solution}
Let $L=1.5\times10^8\text{ cars}$ as before. For cars purchased new, we know
that the average time of ownership, $W_n = 5 \text{ years}$, and for those purchased
used, $W_u = 7 \text{ years}$. Since these times are expected values, we know
that the average time a car spends on the road in total is
\begin{equation}
  W = W_n+W_u=13\text{ years}.
\end{equation}
So then,
\begin{equation}
  \lambda = \frac{L}{W} = \frac{1.5\times10^8\text{ cars}}{13 \text{ years}}
  \approx 1.154\times10^7\text{ cars/year}
\end{equation}

\section{} %3
A graduate program in systems engineering has full time students and part time students.
The number of full-time students who join the program each year is a Poisson random
variable with a mean of 30. The number of part-time students who join the program each
year is a Poisson random variable with a mean of 20. 30\% of full-time students graduate
in 1.5 years and 70\% graduate in 2 years. 50\% of part-time students graduate in 3 years
and 50\% graduate in 4 years.

\subsection*{a}
What is the average number of students enrolled in the program?
\subsubsection*{Solution}
With $L$, $\lambda$, and $W$ as before, apply Little's Law with
\begin{equation}
  \lambda = 20 + 30 = 50\text{ students/year}
\end{equation}
and
\begin{equation}
  W =\frac{0.3(1.5) + 0.7(2) + 0.5(3) + 0.5(4)}{0.3 + 0.7 + 0.5 + 0.5} \text{ (years)}
     \approx 2.675 \text{ years}.
\end{equation}
Then
\begin{equation}
  L = \lambda W = 50\text{ students/year} \times 2.675 \text{ years} \approx 133.8\text{ students}
\end{equation}

\subsection*{b}
On average, at a given moment in time, what fraction of enrolled students are
full-time and what fraction are part-time?

\subsubsection*{Solution}
Since we know the arrival rate, $\lambda_f$, and can calculate the length of program, $W_f$, for
full-time students, we apply Little's law with
\begin{equation}
\lambda_f = 30\text{ students/year}
\end{equation}
and
\begin{equation}
  W_f =0.5(3) + 0.5(4)\text{ (years)}
     = 3.5 \text{ years}.
\end{equation}
Then
\begin{equation}
  L_f = 30\text{ students/year} \times 3.5 \text{ years} = 105\text{ students}
\end{equation}
So the proportion of full-time students is
\begin{equation}
  \frac{L_f}{L}=\frac{105}{133.8} \approx 0.785
\end{equation}
and the proportion of part-time students is
\begin{equation}
  P_p = \frac{L-L_f}{L}=\frac{133.8-105}{133.8} \approx 0.215
\end{equation}

\section{} %4
Customers who have purchased a Delta laptop may call a customer support center to get technical help. Initially, a call is handled by a regular service representative. If the problem cannot be handled by a regular service representative, the call is transferred to a specialist. 20\% of all calls are transferred to a specialist. On average, there are 40
customers being served or waiting to be served by a regular representative. On average,
there are 10 customers being served or waiting to be served by a specialist. The average
rate of incoming calls is 100 per hour. There are 30 regular representatives and 10
specialists.

\subsection*{a}
What is the average time spent in the system for an arbitrary customer? State any
assumptions you make to answer this question.
\subsubsection*{Solution}
We can use the average number, $L_r$, of regular customers and, $L_s$ the number of those
customers needing a specialist to apply Little's Law with $\lambda$ and $W$ as before.
\begin{equation}
  \lambda = 100\text{ customers/hour}
\end{equation}
and
\begin{equation}
  L = L_r + L_s = 40\text{ customers}+10\text{ customers}=50\text{ customers},
\end{equation}
so
\begin{equation}
  W = \frac{L}{\lambda} =\frac{50\text{ customers}}{100\text{ customers/hour}}=0.5\text{ hours}
\end{equation}
\subsection*{b}
What is the average time spent in the system for a customer who needs to talk to
a specialist?
\subsubsection*{Solution}
Since all customers are, at least initially, regular customers, we can
use $\lambda$ to calculate $W_r$, the average time regular customers are on the
line. Then, $W_r$ and $W$ may be used to find $W_s$, the average time customers
that require a specialist are on the line. We know that
\begin{equation}
  L_r = 40\text{ customers}
\end{equation}
and
\begin{equation}
  \lambda_r = \lambda = 100 \text{ customers/hour}.
\end{equation}
Then from Little's law, it follows that
\begin{equation}
  W_r= \frac{L_r}{\lambda_r} =\frac{40\text{ customers}}{100\text{ customers/hour}}
        =0.4\text{ hours}.
\end{equation}
Since
\begin{equation}
  W = \frac{0.2W_s + W_r}{0.2+1},
\end{equation}
we can solve for $W_s$ to find
\begin{equation}
  W_s = \frac{1.2W-W_r}{0.2} = \frac{1.2(0.5)-0.4}{0.2} = 1\text{ hour}.
\end{equation}

\section{} %5
Someone is using a hovering drone to take aerial pictures in a park. If the drone fails
(e.g., a rotor breaks), it might fall anywhere in a circle of some radius (see top view in
figure below). The drone operator is concerned about people who might be hit if the
drone falls. The drone is hovering near several paths. People enter each of the three
points on the circle according to a Poisson process with rate 1 every 5 minutes. Each
person walks at a rate of 1.1 m/s. The path segment lengths are shown in the figure. At
the intersection, each person is equally likely to continue on either of the other two paths.

\subsection*{a}
What is the average number of people at risk to a potential drone failure at any given
time (i.e., what is the average number of people in the circle at any point in time)?
\subsubsection*{Solution}
Within the risk area, there are three paths (two directions each) with an equal
traveler arrival rate of one every five minutes. The total traveler arrival
rate, $\lambda$, is three every five minutes (0.6 travelers/min). The paths
have lengths of 1,200 m, 1500 m, and 1700 m. The average distance is 1,466.67 m.
At a walking speed of 1.1 m/s the average time,$W$, in the risk area is 22.22 min.
The average number of travelers at risk, $L$, is
\begin{align}
  L &= \lambda W \\
    &= 0.6\text{ travelers/min} \times 22.22\text{ min}\approx13.33\text{ travelers}.
\end{align}

\subsection*{b}
Now suppose that 20\% of all people stop at the intersection for 3 minutes to read a
sign before continuing. What is the average number of people exposed in this case?
\subsubsection*{Solution}
Since some travelers (20\%) spend more time at the intersection, the new average time
spent, $W^\prime$, in the risk area is
\begin{equation}
  W^\prime = W + 0.2 \times 3\text{ min} = 22.22\text{ min} + 0.2 \times 3\text{ min}
  = 22.82\text{ min}.
\end{equation}
So the new number of exposed travelers, $L^\prime$, is
\begin{align}
  L^\prime &= \lambda W^\prime \\
    &= 0.6\text{ travelers/min} \times 22.82\text{ min}\approx13.69\text{ travelers}.
\end{align}

\section{} %6
Is it better to split service into separate tasks (a tandem queueing system, like Starbucks
with a cashier and barista) or have each server complete all service tasks (e.g., have 2
servers that do both jobs)? Identify strengths and weaknesses for each option (using
common sense arguments, no math).
\subsubsection*{Solution}
In the Starbucks retail example, we can assume that operations have been
designed to maximize customer throughput while meeting certain threshold
requirements for queue experience quality such as a maximum expected initial wait time limit.

Services often comprise both personnel and mechanisms. In the Starbucks example,
different mechanisms are required for their respective roles such as the cash
register, espresso machine, and blender. Since these devices represent their own
services with customers, service times, and queues, it may be best to match
personnel with mechanisms such that high demand for one device does not
propagate to other queues. There also may be practical reasons to match
personnel with mechanisms such as the skill or authority needed to operate them.
In the Starbucks example, we might reasonably expect all personnel to be
adequately trained to manage all parts of drink preparation, as needed. However,
in more complex production environments, such an arrangement may be impractical.
Keeping personnel at one station may also obviate some tasks altogether.
Cashiers, for example, would need to wash their hands far more frequently if
they operated the cash register and food preparation equipment.

By contrast, some retail food vendors, such as sandwich shops, make products
that require more interaction with the customer to place their order. There, we
often see a server handling one customer's order from start to finish, with a
hand-off only at the cashier. While Starbucks customers may be expected to have
their choices made soon after arriving, sandwich customers may need to be
informed during food preparation what options are available to them based on
their previous choices. This process may be too cumbersome to handle across
multiple servers. This state of affairs is often seen whenever customers must be
informed and make decisions iteratively, e.g., loan officers and travel agents.

There are also trade-offs for customer experience to consider. On one hand, having a
server operate a low latency initial queue such as taking orders at a cash
register may improve customer experience by reducing pre-process wait times.
On the other, the additional interaction and process information of the
sandwich line may do so by reducing uncertain or unoccupied wait times.

\end{document}