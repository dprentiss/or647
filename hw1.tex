\documentclass{amsart}
\usepackage{cancel}
\title{Homework 1}
\author{David Prentiss}

\begin{document}
\maketitle

\section{} %1
A certain football league consists of 32 teams. Each team has 67 active
players. There is a draft each year for teams to acquire new players. Each
team acquires 7 new players per year in the draft. The number of active
players on each team must always be 67. Thus, each team must cut some
existing players each year to make room for the new players.

\subsection*{a}
Assuming that a football player can only join a team by being selected
in the draft, estimate the average career length of a football player in
the league.


\subsubsection*{Solution}

From Little's Law, we know that $L=\lambda W$, where $L$ is the average number
of customers in the system, $\lambda$ is the average rate that customers arrive
to a system, and $W$ is the average time that a customer spends in the system.

For part (a), let the system be a single team. Since the number of active
players per team must be 67, then let $\L = 67 \text{ players}$. Furthermore, since seven
new players are drafted each year, let $\lambda = 7 \text{ players/year}$.

Solving the equation of Little's Law for $W$ we have
\begin{align}
  L &= \lambda W \\
  W &= \frac{L}{\lambda} = \frac{67 \text{ players}}{7 \text{ players/year}} \approx 9.6 \text{ years}.
\end{align}
Note that if we take the system to be the entire league, then
$\L = 67 \text{ players} \times 32 \text{ teams}$
,
$\lambda = 7 \text{ players/year}\times 32 \text{ teams}$
, and we get the same result with
\begin{equation}
  W = \frac{L}{\lambda} = \frac{67 \text{ players} \times \cancel{32 \text{ teams}}}
  {7 \text{ players/year}\times \cancel{32 \text{ teams}}} \approx 9.6 \text{ years}.
\end{equation}

\subsection*{b}
Now, suppose that a player can join a team in one of two ways: (1) by
being selected in the draft, as before, or (2) by signing directly with a
team outside the draft. Suppose further that the average career length of
a football player is known to be 3.5 years. Under the same assumptions
as before, estimate the average number of players who enter the league
each year without being drafted.
\subsubsection*{Solution}
For part (b), we are given $W = 3.5 \text{ years}$. Additionally, the average
number of players joining a team each year is the sum of the seven players
drafted plus the average number of players signing on outside of the draft,
$\lambda_d$, giving $\lambda = 7+\lambda_d$. Solving again, this time for $\lambda_d$, we have
\begin{align}
  L &= \lambda W \\
  L &= (7 + \lambda_d) W \\
  \lambda_d &= \frac{L}{W}-7 = \frac{67 \text{ players}}{3.5 \text{ years}}
              - 7\text{ players/year}\approx 12.1\text{ players/year}.
\end{align}

\section{} %2
The length of time that a person owns a car before buying a new one has an Erlang-3
distribution with a mean of 5 years. Suppose that there are approximately 150 million
cars in the United States.

\subsection*{a}
Assuming that a person's old car is destroyed when he or she buys a new car,
how many cars does the auto industry expect to sell each year?
\subsubsection*{Solution}
Applying Little's Law, let the system be the pool of cars in the United States
with the average number of cars in the system, $L=1.5\times10^8\text{ cars}$, and the
average time spent in the system, $W=5 \text{ years}$. Then, solving for the
average rate of cars entering the the system, $\lambda$, we have
\begin{align}
  L &= \lambda W \\
  \lambda &= \frac{L}{W} = \frac{1.5\times10^8\text{ cars}}{5 \text{ years}} = 3\times10^8\text{ cars/year}
\end{align}
\subsection*{b}
Now assume that a person's old car is sold to somebody else when that person
buys a new car. The person who buys the used car keeps it for period of time
following an Erlang-3 distribution with a mean of 7 years. When that person buys
another used car, his or her previous used car is assumed to be destroyed. Under
the same previous assumptions, how many new cars does the auto industry
expect to sell each year?
\subsubsection*{Solution}
Let $L=1.5\times10^8\text{ cars}$ as before. For cars purchased new, we know
that the average time of ownership, $W_n = 5 \text{ years}$, and for those purchased
used, $W_u = 7 \text{ years}$. Since these times are expected values, we know
that the average time a car spends on the road in total is
\begin{equation}
  W = W_n+W_u=13\text{ years}.
\end{equation}
So then,
\begin{equation}
  \lambda = \frac{L}{W} = \frac{1.5\times10^8\text{ cars}}{13 \text{ years}}
  = 1.154\times10^7\text{ cars/year}
\end{equation}

\section{} %3
A graduate program in systems engineering has full time students and part time students.
The number of full-time students who join the program each year is a Poisson random
variable with a mean of 30. The number of part-time students who join the program each
year is a Poisson random variable with a mean of 20. 30\% of full-time students graduate
in 1.5 years and 70\% graduate in 2 years. 50\% of part-time students graduate in 3 years
and 50\% graduate in 4 years.

\subsection*{a}
What is the average number of students enrolled in the program?
\begin{align}
\lambda &= 20 + 30 = 50\text{ students/year}\\
  W &=\frac{0.3(1.5) + 0.7(2) + 0.5(3) + 0.5(4)}{0.3 + 0.7 + 0.5 + 0.5} \text{ (years)}
     = 2.675 \text{ years} \\
  L &= 50\text{ students/year} \times 2.675 \text{ years} = 133.8\text{ students}
\end{align}
\subsubsection*{Solution}

\subsection*{b}
On average, at a given moment in time, what fraction of enrolled students are
full-time and what fraction are part-time?

\subsubsection*{Solution}
\begin{align}
\lambda_f &= 30\text{ students/year}\\
  W_f &=0.5(3) + 0.5(4)\text{ (years)}
     = 3.5 \text{ years} \\
  L_f &= 30\text{ students/year} \times 3.5 \text{ years} = 105\text{ students}\\
  P_f &= \frac{L_f}{L}=\frac{105}{133.8} = 0.785\\
  P_p &= \frac{L-L_f}{L}=\frac{133.8-105}{133.8} = 0.215
\end{align}

\section{} %4
Customers who have purchased a Delta laptop may call a customer support center to get
technical help. Initially, a call is handled by a regular service representative. If the
problem cannot be handled by a regular service representative, the call is transferred to a
specialist. 20\% of all calls are transferred to a specialist. On average, there are 40
customers being served or waiting to be served by a regular representative. On average,
there are 10 customers being served or waiting to be served by a specialist. The average
rate of incoming calls is 100 per hour. There are 30 regular representatives and 10
specialists.

a. What is the average time spent in the system for an arbitrary customer? State any
assumptions you make to answer this question.

b. What is the average time spent in the system for a customer who needs to talk to
a specialist?
\subsubsection*{Solution}
\section{} %5
Someone is using a hovering drone to take aerial pictures in a park. If the drone fails
(e.g., a rotor breaks), it might fall anywhere in a circle of some radius (see top view in
figure below). The drone operator is concerned about people who might be hit if the
drone falls. The drone is hovering near several paths. People enter each of the three
points on the circle according to a Poisson process with rate 1 every 5 minutes. Each
person walks at a rate of 1.1 m/s. The path segment lengths are shown in the figure. At
the intersection, each person is equally likely to continue on either of the other two paths.
a. What is the average number of people at risk to a potential drone failure at any given
time (i.e., what is the average number of people in the circle at any point in time)?
b. Now suppose that 20\% of all people stop at the intersection for 3 minutes to read a
sign before continuing. What is the average number of people exposed in this case?
\subsubsection*{Solution}
\section{} %6
Is it better to split service into separate tasks (a tandem queueing system, like Starbucks
with a cashier and barista) or have each server complete all service tasks (e.g., have 2
servers that do both jobs)? Identify strengths and weaknesses for each option (using
common sense arguments, no math).
\subsubsection*{Solution}
\end{document}